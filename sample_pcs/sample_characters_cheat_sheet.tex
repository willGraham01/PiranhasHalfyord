\documentclass[10pt,a4paper]{article}
\usepackage[utf8]{inputenc}
\usepackage{amsmath}
\usepackage{amsfonts}
\usepackage{amssymb}
\usepackage{graphicx}
\usepackage{parskip}

\title{Sample Characters: How they Play}
\author{Will Graham}

\begin{document}

\maketitle

\tableofcontents

\section{Overview}

This document is intended to provide some guidance on the sample player characters that are provided with the adventure.
The sample characters are provided with the intention that new players can be given a character that is set up to be reasonably useful both in and outside the combat encounters that the story permits.
The characters also each have a backstory hook that new players can use as a prompt as to why the character finds themselves in Halfyoord at the start of the adventure, but they are by no means required to use this if they want to adapt the backstory.
On a related note, the sample characters are deliberately given no "deep" backstory for this reason, and players should also feel welcome to change some of the preset physical attributes that are given.

For each character, their corresponding section briefly outlines:
\begin{itemize}
    \item Their class and race
    \item The key features of their class
    \item A brief outline of the default reason for the character being in Halfyoord
    \item Their out-of-combat utility
    \item Their in-combat utility
\end{itemize}

Players are of course free to ignore this guidance, but for new players it should hopefully give some idea as to what their character is meant to be good at, and the role(s) they can fulfil well.
It can also provide some inspiration for backstory expansion.

The document ends with another "combat cheat sheet", which the individual entries for each of the sample characters refer to in places.
This is intended to streamline combat for new players, and help give them a plan of action for their turn when it comes.
Of course, it should be emphasised that certain combats or encounters might call for out-of-the-box thinking!

\newpage

\section{How they play: The Rogue}

The sample rogue is designed to be an investigator; their sample backstory is that they've been hired by the Ordo Magicae (a guild for magic users) to investigate the disappearance of one of their members in Halfyoord.
They are fairly young for a (wood) elf, and it is usually quite odd to see such an elf away from a forest community.

Your skills primarily revolve around uncovering hidden information, either through subterfuge or by reading people.
You're particularly good at:
\begin{itemize}
    \item Using your \emph{thieves' tools} to pick locks, and gaining access to areas your might not have access to undetected (\emph{stealth}).
    \item \emph{Investigating} an area for clues or hidden details.
    \item Reading people's body language to see if they're lying (\emph{Insight, Ear for Deceit ability}).
    \item Being aware of your surroundings and picking out details (\emph{Perception}).
    \item You are also quite knowledgeable of current and historical events (\emph{History}).
\end{itemize}

Most of your other abilities try to complement these strengths.

\subsection{In Combat}

You aren't the toughest fighter, and typically you'll want to be attacking from range with your shortbow - though you do have a rapier (sword) if you need to get up close and personal.
Your fighting style should revolve around your \emph{cunning action} ability, and your \emph{sneak attack} - on most turns you'll want to take the attack action.

Sneak attack lets you deal extra damage when you hit a distracted enemy with an attack.
The ability on your character sheet describes how you can trigger this extra damage:
\begin{itemize}
    \item Target enemies who are already fighting your friends in melee.
    \item If you are hidden after taking the \emph{Hide} action, you will get advantage when you attack someone, which also triggers your sneak attack.
    \item You also have the \emph{Insightful Fighting} ability if there is no other way for you to apply your sneak attack.
\end{itemize}

Due to your \emph{cunning action} ability, you have a lot of things you can do with your bonus action.
For example, you could disengage using a \emph{bonus action} if an enemy gets too close to you - and then use your \emph{action} to attack them with your bow after moving away.
You can also \emph{hide} as a bonus action - so you might first try to hide from an enemy with your bonus action, and then the same turn attack them after hiding to get your sneak attack!
You could instead \emph{hide} after attacking someone so that they don't try and find you on their turn and take revenge!

\textbf{TLDR}; try to stay at range, get sneak attack as often as you can.

\newpage

\section{How they play: The Sorcerer}

The sample sorcerer provides some insight into the magical aspects of the world, and is something of a "glass cannon" in a fight - capable of devastating magical attacks but very vulnerable to melee attacks.
Their sample backstory is that they were educated by the Ordo Magicae after their latent magical abilities began to emerge, and have since found gainful employment working with the guild.
They are in Halfyoord on an errand to check in with the status of the guild's outpost there.

Possessing dragon ancestry makes you quite \emph{intimidating} to most people - but be careful about who you try and scare.
Your natural magical abilities also give you good insights into the nature of magic (\emph{Arcana}) and the divine forces (\emph{Religion}).

You also have a few spells that might be useful outside of combat.
\emph{Light} is always useful to illuminate dark areas, and \emph{Minor Illusion} can be good for creating distractions - both are cantrips so you can cast them as many times as you like.
You also have \emph{Charm Person} if you need to sweet talk someone, and \emph{Detect Magic} if you want to check for magical traps or identify something you know is magical.

\subsection{In Combat}

Whilst you have a decent amount of HP, you are very easy to hit, so you'll want to try and stay away from enemies.
Instead, focus on using your spells to do damage or impede your enemies - your \emph{Metamagic} ability goes a long way to making your spells more powerful.

On most of your turns, you'll want to cast a spell - it is generally not worth you using your melee weapons unless you really have no other option!
\emph{Firebolt} lets you hit enemies from range, whilst \emph{Shocking Grasp} lets you hit enemies that get too close to you.
Both of them are cantrips, so they don't cost you anything to cast.

If you're looking for area-of-effect damage, you can use \emph{Shatter}.
\emph{Web} is a great spell for slowing enemies down - preventing them from reaching you to attack, or from running away as quickly.
You've also got your \emph{Breath Weapon} for some rather nice damage, but can only use it once a day.

You don't have any bonus actions, but your \emph{Metamagic} ability can be used to turn one of your spells into a \emph{bonus action}.
This means you can do something else with your action, such as hide, disengage is the enemies are too close, or even cast another spell!
You can also use this ability to extend the range of one of your spells, if an enemy is just too far away.
Keep in mind these things use up your \emph{Scorcery points} though - and they only come back after you rest.

\newpage

\section{Appendix: Combat Quick Reference}

\subsection{Roll for Initiative!}

Combat starts with everyone involved (the party, \emph{and} the enemies!) \textbf{rolling for initiative}.
To roll for initiative, roll a d20 and add your initiative modifier - it should be near the top-and-centre of the first page of your character sheet.
Let the DM know the result - higher is better!

Once everyone's result is in, the values determine the turn order for combat.
Going from highest initiative to lowest, each participant in combat will take their turn, looping back to the top of the order once everyone has gone, until combat ends.
\emph{One round of combat} refers to the above - going all the way through the initiative order until you get back to the person who started.

Each round of combat lasts 6-in-game-seconds.
This is important because some spells in the game only last for one minute - this translates to 10 rounds in combat, meaning ``you'll get 10 turns of this effect in a battle".
For context, most fights in DnD are resolved in less than 4 rounds.

\subsection{What can I do in Combat?}

On each round of combat, you (or rather, your character) has:
\begin{itemize}
    \item \emph{Movement}, which you take on your turn.
    \item One \emph{action}, which you take on your turn.
    \item One \emph{bonus action}, which you also take on your turn.
    \item One \emph{reaction}, which you take on someone else's turn, usually in response to something they do.
\end{itemize}

\subsubsection{On your turn}

As mentioned above, on your turn you have movement, one action, and one bonus action.
You can do these in any order, and you can break them up / interleave them too: you could move half your movement, take an action, then move another 5 feet, take a bonus action, then move the rest of your distance if you wanted.

So the next question is, what can I do with these things?
Movement is hopefully clear - it just dictates how far you can move this turn.
Your character will have a \emph{speed} (usually in a box next to your initiative modifier) which is how many feet they can move.
You can spend as much of this as you like on your turn.
Melee fighters will typically use it to get close to the action - ranged characters and magic users will typically be backing away!

The main part of your turn will be your \textbf{action}.
There is a full list of actions you can take in the Player's Handbook, but the most commonly used ones are:
\begin{itemize}
    \item Attack: Use a weapon (sword, mace, bow) to make attacks against another combatant. You make one attack - some classes let you make more.
    \item Cast a spell: If your character has access to spells, they undoubtedly have a few spells that do damage! But they also might have a spell that restrains an enemy, or puts them to sleep, or heals an ally. Any spell that has a ``casting time" of 1 action can be cast on your turn.
    \item Dash: Double your movement speed for this turn.
    \item Hide: If your character is suitably concealed, they can attempt to hide from enemies who might be pursuing them.
    \item Disengage: Your character becomes immune to \emph{attacks of opportunity} from other characters. See ``reactions" below.
    \item Ready: Delay your turn until later. See ``reactions" below.
\end{itemize}

What you try and do isn't limited to the actions above.
If you want to attempt something in combat that doesn't fit into a box above, like maybe try and wrestle an important item from a thief's hands, or knock a stack of boxes over to block someone's path, just ask the DM.
They will rule on whether it's possible, and what you'll need to do.

The only other thing you get on your turn is a \emph{bonus action}, and these typically only come from your class or race.
Some spells have casting times of ``1 bonus action" - these spells can be cast using your bonus action, just like regular spells with the ``cast a spell" \emph{action}.
However, you cannot cast two spells on the same turn, unless one of them is a cantrip.

\subsubsection{Reactions}

\textbf{Reaction}s are responses to events that might happen in combat, and are usually only available through certain spells or abilities that your class / race gives you.
For example, there is a spell called ``Shield" which has a casting time of ``1 reaction" - this means that a player can use their reaction to cast the spell.
The most common reactions (that every character can make use of) are:
\begin{itemize}
    \item \emph{Attack of Opportunity}: If an enemy who is in your melee range uses their movement to leave your melee range, you can use your reaction to make a melee attack against them. This represents them trying to run away, opening up their defences and giving you a chance to strike!
    \item \emph{Ready an Action}: You can ``delay" your turn in combat, and use your reaction to take your turn later. This is quite useful if you are waiting for an enemy to come out of hiding, so you can hit them when they pop up before trying to run away.
\end{itemize}

\end{document}